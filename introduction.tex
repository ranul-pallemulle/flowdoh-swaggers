\section{Introduction}
\subsection{Motivation}
Flowdoh has been improving steadily in its ability to connect with and use external APIs. Examples of such integrations include the Microsoft Outlook API and the Google Cloud Translations API. This is made possible by the use of Swagger files. In order to be able to connect any external API of choice it is necessary to understand how Flowdoh uses Swagger files and how to write them. In this document we will explore Swagger files and how to structure them for use with Flowdoh.

\subsection{Background}
A Swagger file is a text file that defines an API. It can be written in JSON format or YAML format. It contains such information as the API host (the URL to interact with), the authentication method(s) to use, the different operations that are available, and much more. What makes this file a "Swagger" file is the structure of the data within it. The structure is defined by the OpenAPI Specification\cite{openapispec} (formerly the Swagger specification). In essence, the goal of the specification is to provide a document format that is sufficiently generic such that it can define the majority of APIs that are available. The format should be expressive enough to be human-readable, and it should be precise enough for a computer to parse and gather all the information needed to interact with the API. As a result of improvements on both fronts, several versions of the OpenAPI specification have emerged. There are two major versions - 2.0 and 3.0. Version 3.0 has several patch versions, such as 3.0.1 and 3.0.2.

OpenAPI 3.0 overcomes many of the limitations of version 2.0 \cite{openapivsswagger}. Any API that can be defined using OpenAPI 2.0 can also be defined using OpenAPI 3.0, perhaps even with greater precision. Moreover, certain features in Flowdoh, such as Triggers from external applications, can only be defined using OpenAPI 3.0. As such, we will limit the discussion in this document to OpenAPI 3.0.

All current versions of Flowdoh require Swagger files to be defined in JSON format. Therefore, all examples in this document will be in JSON format. However, it is not a difficult task to convert a JSON-formatted Swagger file to YAML or vice-versa.