\section{Writing a Swagger File}
\subsection{A minimal Swagger file}
Listing \ref{code:minimum_swagger} shows all the required fields as defined by the OpenAPI specification. Note that a minimally valid Swagger file according the specification is not a valid Flowdoh Swagger file. There are certain fields in the Swagger file that, despite being defined as optional in the OpenAPI specification, are required by Flowdoh to be present. For example, Flowdoh requires the \texttt{servers} section to be defined so that the base URL of the API is known.
\begin{lstlisting}[caption={A minimal OpenAPI 3.0 definition},label={code:minimum_swagger},language=json]
{
    "openapi": "3.0.2",
    "info": {
        "title": "A minimal Swagger file",
        "version": "1.0"
    },
    "paths": {

    }
}
\end{lstlisting}
Note that the \texttt{paths} object is required but may be left empty. According to the specification, this is to allow for access control \cite{empty_paths} - a person may be able to access the Swagger file but not have access to the available paths. Flowdoh requires that Swagger files have a populated \texttt{paths} section as it expects to discover operations that can be invoked on those paths.
\subsection{The \texttt{openapi} section}
This field defines the version of the OpenAPI specification that the file conforms to. It is only valid in OpenAPI 3.0 files. If the file conforms to the OpenAPI 2.0 specification, it should instead have a \texttt{swagger} section as shown in listing \ref{code:swagger_2.0}.
\begin{lstlisting}[caption={Alternative to the \texttt{openapi} section in OpenAPI 2.0},label={code:swagger_2.0},language=json]
{
    "swagger": "2.0"
}
\end{lstlisting}
Flowdoh only checks for the first (major) portion of the version number to identify whether the file should be parsed as OpenAPI 2.0 or 3.0. Therefore, any minor version number for OpenAPI 3.0 is regarded as valid. For example, the version number could be indicated as "3.0.9" even though such a version of the specification does not exist. Flowdoh will parse any OpenAPI 3.0 file as if it were of version 3.0.2 as that is the latest officially supported version in Flowdoh. Given that the OpenAPI 3 specification is designed to be backwards compatible, Flowdoh is able to parse all versions of OpenAPI 3.0 documents, but will ignore any features that are added in versions 3.0.3 and later.
\subsection{The \texttt{info} section}
\begin{lstlisting}[caption={A fully populated \texttt{info} section},label={code:info_section},language=json]
{
    "info": {
        "title": "The Food API",
        "description": "This file describes a fictional API that provides details about food items",
        "termsOfService": "https://myfoodapi.com/termsofservice",
        "contact": {
            "name": "Food API Support",
            "url": "https://myfoodapi.com/support",
            "email": "support@myfoodapi.com"
        },
        "license": {
            "name": "GPLv3",
            "url": "https://myfoodapi.com/LICENSE.html"
        },
        "version": "v15.0"
    }
}
\end{lstlisting}
This section is used to add some metadata about the API. The \texttt{title} field gives a name to the API or the Swagger file. The \texttt{version} field provides the version number of the API that is being described. Other information such as contact details for support, terms of service and license details may be optionally provided. Flowdoh looks for the \texttt{title} and \texttt{version} fields in this section, but only for the purposes of saving the app. No actual functionality in Flowdoh makes use of the \texttt{info} section. Nevertheless, it should still be included in any Swagger file given it is a required field and provides some benefit to the human reader in identifying which API the file describes.
\subsection{The \texttt{servers} section}

\begin{lstlisting}[caption={A URL defined in the \texttt{servers} section},label={code:servers},language=json]
{
    "servers": [
        {
            "url": "https://myapi.com"
        }
    ]
}
\end{lstlisting}
This section defines the base URLs for the API. It is possible to include multiple URLs here. Currently, Flowdoh only uses the first one listed. The OpenAPI specification allows for the use of \texttt{server variables} which can be used for substituting values in the url. This feature is not currently supported in Flowdoh. The specification also allows a server url to be a relative path which would indicate that the URL is relative to the location of the Swagger file itself. This too is not supported by Flowdoh - it is expected that the server URL is an absolute URL.
\subsection{The \texttt{paths} section}
This is the section within which all operations that can be called against the API are defined. Each \texttt{path} in this section can be appended to the host URL to produce a URL that corresponds to the resource being described.
\begin{lstlisting}[caption={The \texttt{paths} object},label={code:paths},language=json]
{
    "paths": {
        "/api/v5/users": {
            "get": {
                "summary": "Get list of users",
                "operationId": "getListOfUsers",
                "responses": {
                    "200": {
                        "description": "Successful retrieval of users",
                        "content": {
                            "application/json": {
                                "schema": {
                                    "$ref": "#/components/schemas/UsersList"
                                }
                            }
                        }
                    }
                }
            }
        }
    }
}
\end{lstlisting}
Listing \ref{code:paths} shows the \texttt{paths} object populated with a single \texttt{path item} corresponding to the server path \texttt{/api/v5/users}. If the \texttt{servers} section of the Swagger file were to contain a URL such as in listing \ref{code:servers}, it could be deduced that the absolute URL for this path is \texttt{https://myapi.com/api/v5/users}. That is, the path is always appended to the server URL to obtain the absolute URL to the resource. There can be many paths defined in the \texttt{paths} section. For example, there may be a different path with the relative URL \texttt{/api/v5/users/me} with a similar \texttt{get} operation defined within it which may be used for getting details of the currently authenticated user.
\subsubsection{Operations}
\label{subsubsec:operations}
Just as there can be many paths defined within the \texttt{paths} section, there can be many operations defined within a single path item. Operations correspond to the HTTP methods \texttt{GET}, \texttt{POST}, \texttt{PUT}, \texttt{DELETE}, \texttt{PATCH}, \texttt{HEAD}, \texttt{OPTIONS} and \texttt{TRACE}. The combination of a path and method correspond to a single operation that can be carried out against the API. Flowdoh currently supports the HTTP methods \texttt{GET}, \texttt{POST}, \texttt{PUT}, \texttt{DELETE} and \texttt{PATCH}.

An operation may require some parameters to be provided before it can be invoked. In a HTTP request, parameters may be provided as part of the URL, in the request body, in cookies, and in the request headers. URL parameters may be in the form of path (route) parameters or query parameters. The OpenAPI specification allows for such parameters to be defined as part of an operation.
\begin{lstlisting}[caption={A \texttt{POST} request with parameters},label={code:operationwithparams},language=json]
{
    "paths": {
        "/api/v5/{organizationId}/users": {
            "post": {
                "summary": "Create a new user in an organization",
                "operationId": "createUser",
                "parameters": [
                    {
                        "name": "organizationId",
                        "in": "path",
                        "required": true,
                        "schema": {
                            "type": "integer",
                            "format": "int32"
                        }
                    },
                    {
                        "name": "sendEmail",
                        "in": "query",
                        "required": false,
                        "schema": {
                            "type": "boolean",
                            "default": false
                        }
                    }
                ],
                "requestBody": {
                    "required": true,
                    "content": {
                        "application/json": {
                            "schema": {
                                "$ref": "#/components/schemas/User"
                            }
                        }
                    }
                },
                "responses": {
                    "201": {
                        "description": "Successfully created the user"
                    }
                }
            }
        }
    }
}
\end{lstlisting}
Listing \ref{code:operationwithparams} shows an operation that takes path, query, and body parameters. The path URL makes use of path templating in order to define path parameters. When the operation is invoked, \texttt{\{organizationId\}} is replaced using the value provided by the user, and a query string is appended to the URL using the value provided for the \texttt{sendEmail} parameter. Refer to the \hyperref[sec:actions]{Actions section} of this document to see how Flowdoh provides the ability for users to input values for parameters.

Parameters may also be provided at the path level rather than the operation level. These parameters then apply to all operations defined within the path.

The \texttt{responses} object is a required field. It defines the possible responses of the API to the invocation of the operation. A given response may or may not contain data, but will have an associated response code, for example \texttt{201} in listing \ref{code:operationwithparams}. \texttt{2xx} response codes indicate a successful operation and Flowdoh uses such a code as an indication that the operation completed without any errors. If a different code such as one in the range \texttt{4xx} or \texttt{5xx} is encountered, Flowdoh assumes the operation failed and will not proceed with the workflow. The keyword \texttt{default} may be used in the Swagger file to describe responses that are not specifically listed under a response code. Flowdoh treats such a definition as that of a failed operation (i.e. similar to a \texttt{4xx} or \texttt{5xx} response). In other words, if data from the API's response is to be used within the workflow, the response must be defined under a \texttt{2xx} code in the Swagger file.

The OpenAPI specification allows for URL, cookie, and header parameters to contain a \texttt{content} property instead of a \texttt{schema} property for more complex scenarios. This is not supported by Flowdoh; Flowdoh always expects a \texttt{parameter} to contain a \texttt{schema} object. Flowdoh also does not currently support cookie parameters. Flowdoh does not permit URL and header parameters to be of a non-primitive type. That is, these parameters must be of type \texttt{string}, \texttt{integer}, \texttt{number}, or \texttt{boolean}, and cannot be of type \texttt{object} or \texttt{array}. However, Flowdoh permits the \texttt{requestBody} schema to contain any valid OpenAPI 3.0 data type.

Flowdoh is able to handle JSON arrays in the response or request body as long as they are defined as the JSON root, or are defined within a JSON object. Flowdoh is currently unable to handle a situation where an array is defined within another array.

Flowdoh also allows aggregate schemas to be defined using the \texttt{allOf}, \texttt{anyOf}, and \texttt{oneOf} keywords. The keyword \texttt{not} is not currently supported.

\subsection{The \texttt{components} section}
This section contains reusable objects that can be referenced from other parts of the Swagger file by using the \texttt{\$ref} keyword. For example, listing \ref{code:components} shows the \texttt{UsersList} and \texttt{User} schemas referenced by the operations in listings \ref{code:operationwithparams} and \ref{code:paths}. Items inside the components section can also reference other items in the components section.

\begin{lstlisting}[caption={Components section with schemas},label={code:components},language=json]
{
    "components": {
        "schemas": {
            "UsersList": {
                "type": "array",
                "items": {
                    "$ref": "#/components/schemas/User"
                }
            },
            "User": {
                "type": "object",
                "properties": {
                    "firstName": {
                        "type": "string"
                    },
                    "lastName": {
                        "type": "string"
                    },
                    "age": {
                        "type": "integer",
                        "format": "int32"
                    },
                    "address": {
                        "type": "object",
                        "properties": {
                            "city": {
                                "type": "string"
                            },
                            "country": {
                                "type": "string"
                            }
                        }
                    }
                }
            }
        }
    }
}
\end{lstlisting}
The components section is only necessary if other sections have references to it. The \texttt{UsersList} and \texttt{User} schemas could alternatively have been defined in-line, in which case the \texttt{components} section could be omitted, provided there are no other references to it. The \texttt{components} section can contain other types of reusable objects as well such as \texttt{parameters} and \texttt{responses}.

\subsection{Security}
The OpenAPI specification allows for the available authentication methods to be defined in the Swagger file. It also allows the application of these methods to all operations, or specific operations only. Flowdoh currently does not use any authentication information defined in the Swagger file. Instead, the user must manually input all authentication details during app registration. Furthermore, Flowdoh applies the same authentication method and access level requirements to all available operations in the Swagger file.

\begin{lstlisting}[caption={OAuth2 authorization code flow defined in the Swagger file},label={code:security},language=json]
{
    "paths": {
        "/api/send": {
            "post": {
                "security": {
                    "myApiAuth": [
                        "api:read",
                        "api:write"
                    ]
                }
            }
        }
    },
    "security": {
        "myApiAuth": [
            "api:read"
        ]
    }
    "components": {
        "securitySchemes": {
            "myApiAuth": {
                "type": "oauth2",
                "description": "OAuth2 authorization for my API",
                "flows": {
                    "authorizationCode": {
                        "authorizationUrl": "https://auth.myapi.com/authorize",
                        "tokenUrl": "https://auth.myapi.com/token",
                        "refreshUrl": "https://auth.myapi.com/token",
                        "scopes": {
                            "api:read": "Grants access to read data",
                            "api:write": "Grants access to write data"
                        }
                    }
                }
            }
        }
    }
}
\end{lstlisting}
Listing \ref{code:security} shows how OAuth2 authorization with different scopes may be defined in the Swagger file. In this example are two scopes available - \texttt{api:read} and \texttt{api:write}. The \texttt{myApiAuth} authentication method is applied at the root \texttt{security} object with scope \texttt{api:read}. This means that all operations defined in the Swagger file require the user to have permission to read data, but does not require the user to have permission to write data. This requirement is overriden by the \texttt{/api/send} \texttt{POST} operation which requires elevated priviledges - the user must have permission to both read from and write data to the API. Given that Flowdoh does not use security information from the Swagger file and that the same permissions are requested for all operations, all possible scopes must be requested during app registration. That is, both \texttt{api:read} and \texttt{api:write} scopes must be specified, even if most endpoints only need one of the scopes or no authentication at all. For other authentication methods, the highest access level required must be requested.