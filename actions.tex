\section{Flowdoh Actions}
\label{sec:actions}
In this section we will focus on topics that are more geared towards correctly defining Flowdoh Actions and how to use specific properties in the Swagger file to achieve some desired effect.
\subsection{Defining Actions}
By default, Flowdoh recognizes an \texttt{operation} within a given \texttt{path} in the Swagger file as an Action. While the OpenAPI specification allows for a large number of media types\cite{mediatypes} to be specified for operation parameters and responses, Flowdoh currently only supports the media types \texttt{application/json}, \texttt{application/octet-stream}, \texttt{application/x-www-form-urlencoded} and \texttt{multipart/form-data} for use with Actions and Triggers. Actions also support common binary media types in responses (see section on Binary data for more details).

Action parameters and responses are defined in the Swagger file as described in the \hyperref[subsubsec:operations]{Operations} section of this document. Parameters for which values aren't provided by the user will not be sent as part of the operation if they aren't required parameters and no default value is defined in the Swagger file. If they are required parameters, a suitable empty value will be sent. For example, if the parameter type is \texttt{string} the value \texttt{""} will be sent.

In addition, several custom properties, also known as Swagger extensions, can be defined. Extensions are application-specific properties that are not governed by the OpenAPI specification\cite{extensions}. Flowdoh-specific extensions have names that are prefixed with \texttt{x-zen-} and are summarised in Table \ref{table:extensions}.
\begin{table}[h!]
\centering
    \begin{tabularx}{0.9\textwidth}{ X X }
        \hline
        Extension & Summary \\
        \hline
        \texttt{x-zen-action} & Alternative to \texttt{operationId} as a name for the action. May contain spaces. \\
        \hline
        \texttt{x-zen-trigger} & Similar to \texttt{x-zen-action} but also identifies the operation as a Trigger and not an Action. \\
        \hline
        \texttt{x-zen-display-name} & Alternative display name for a parameter or context item. \\
        \hline
        \texttt{x-zen-hidden-input} & Prevents a parameter from being displayed to the user in the Action configuration. \\
        \hline
        \texttt{x-zen-callback-url} & Identifies the callback URL parameter in the Trigger webhook request. \\
        \hline
        \texttt{x-zen-array-index} & Selects a specific index of an array in an Action/Trigger response, rather than using the entire array. \\
        \hline
        \texttt{x-zen-hook-renew-period} & Defines a time duration (in minutes) after which a Trigger webhook should be automatically re-established. \\
        \hline
    \end{tabularx}
    \caption{List of Flowdoh Swagger extensions that apply to Actions and/or Triggers}
    \label{table:extensions}
\end{table}
\subsection{The \texttt{x-zen-action} property}
The \texttt{operationId} property is used to uniquely identify an operation within the Swagger file. If the \texttt{x-zen-action} property is not present, the value of the \texttt{operationId} field will be displayed in the Action selection list when inserting a new Action into a workflow. \texttt{operationId} is typically supplied as a name without spaces - e.g. \texttt{GetUserById}. Therefore, it might be a good idea to provide a separate name for the action for display purposes. The displayed name may be overridden by specifying the \texttt{x-zen-action} property. Note that even if the \texttt{x-zen-action} property is specified, it is recommended that the \texttt{operationId} property is still supplied as it may be used for other purposes such as linking operations together (see section on \hyperref[sec:triggers]{Triggers}).\\
\begin{minipage}{\textwidth}
\begin{lstlisting}[caption={\texttt{x-zen-action} usage},label={code:x-zen-action},language=json]
{
    "paths": {
        "/api/v5/users": {
            "get": {
                "summary": "Get list of users",
                "operationId": "getListOfUsers",
                "x-zen-action": "Get List of Users"
            }
        }
    }
}
\end{lstlisting}
\end{minipage}

\subsection{The \texttt{x-zen-display-name} property}
Similar to the \texttt{x-zen-action} property, this property provides an alternative display name to a field defined inside a \texttt{schema} property. It is applicable to parameters as well as responses. For parameters, the custom values are displayed in the Action configuration panel. For responses, the custom values are displayed in the context selection panel.\\
\begin{minipage}{\textwidth}
\begin{lstlisting}[caption={\texttt{x-zen-display-name} usage},label={code:x-zen-display-name},language=json]
"requestBody": {
    "content": {
        "application/json": {
            "schema": {
                "type": "object",
                "properties": {
                    "firstName": {
                        "type": "string",
                        "x-zen-display-name": "First Name"
                    },
                    "address": {
                        "type": "object",
                        "properties": {
                            "city": {
                                "type": "string",
                                "x-zen-display-name": "City of Residence"
                            },
                            "country": {
                                "type": "string",
                                "x-zen-display-name": "Country of Residence"
                            }
                        }
                    }
                }
            }
        }
    }
}
\end{lstlisting}
\end{minipage}
\subsection{The \texttt{x-zen-array-index} property}
Some actions (and triggers) return data that contain JSON arrays. In Flowdoh, data inside arrays can only be accessed in the workflow context by using a loop which will do some processing for each element of the array. In some cases it may be known that only a single element of the array is useful to the workflow. This property may then be used to extract the data out of the array so that it can be used in the workflow context without the use of loops.\\
\begin{minipage}{\textwidth}
\begin{lstlisting}[caption={\texttt{x-zen-array-index} usage},label={code:x-zen-array-index},language=json]
"responses": {
    "200": {
        "description": "Success",
        "content": {
            "application/json": {
                "schema": {
                    "type": "object",
                    "properties": {
                        "value": {
                            "type": "array",
                            "x-zen-array-index": 0,
                            "items": {
                                "type": "object",
                                "properties": {
                                    "messageId": {
                                        "type": "string"
                                    },
                                    "messageContent": {
                                        "type": "string"
                                    }
                                }
                            }
                        }
                    }
                }
            }
        }
    }
}
\end{lstlisting}
\end{minipage}
In listing \ref{code:x-zen-array-index}, if the \texttt{x-zen-array-index} property were not specified, the \texttt{value} property must be mapped to a loop start node before the properties \texttt{messageId} and \texttt{messageContent} are available as context data (and even then they will only be available inside the loop). The inclusion of \texttt{x-zen-array-index} indicates to Flowdoh that only the first item of the array (index 0) should be used and not the entire array. This allows \texttt{messageId} and \texttt{messageContent} to be available in the context just like any other context item.
\subsection{Binary data}
A user can use Actions to send binary data such as documents and images to external APIs. Binary data can also be received from an external API as part of the action response or trigger. This can be achieved by using the media types \texttt{application/json} or \texttt{multipart/form-data} in the \texttt{requestBody} or \texttt{responses} section of an operation. In the case of actions that return plain binary data, media types such as \texttt{image/jpeg}, \texttt{image/png}, or \texttt{application/pdf} can be used. If the action may return various file types, the generic media type \texttt{application/octet-stream} may be used. To identify plain binary data in the workflow context, use the \texttt{x-zen-display-name} property (see listing \ref{code:binaryinresponse}). In the \texttt{requestBody}, if \texttt{application/json} is used, the file will be sent as a base64-encoded string. Else, if \texttt{multipart/form-data} is used, the file will be sent as raw binary data. To identify an item as binary content within the Swagger file, it must be given the type \texttt{string}. If the media type is \texttt{application/json}, the item must be given the format \texttt{byte}. If the media type is \texttt{multipart/form-data}, or is a different file media type such as \texttt{application/pdf}, the item must be given the format \texttt{binary}.\\
\begin{minipage}{\textwidth}
\begin{lstlisting}[caption={Defining files in JSON data},label={code:binaryinjson},language=json]
"requestBody": {
    "content": {
        "application/json": {
            "schema": {
                "type": "object",
                "properties": {
                    "fullName": {
                        "type": "string"
                    },
                    "address": {
                        "type": "string"
                    },
                    "resume": { // binary data
                        "type": "string",
                        "format": "byte"
                    }
                }
            }
        }
    }
}
\end{lstlisting}
\end{minipage}
\begin{minipage}{\textwidth}
\begin{lstlisting}[caption={Defining files in a form post},label={code:binaryinmultipart},language=json]
"requestBody": {
    "content": {
        "multipart/form-data": {
            "schema": {
                "type": "object",
                "properties": {
                    "fullName": {
                        "type": "string"
                    },
                    "address": {
                        "type": "string"
                    },
                    "resume": { // binary data
                        "type": "string",
                        "format": "binary"
                    }
                }
            }
        }
    }
}
\end{lstlisting}
\end{minipage}
\begin{minipage}{\textwidth}
\begin{lstlisting}[caption={Response returning a file},label={code:binaryinresponse},language=json]
"responses": {
    "200": {
        "description": "A pdf or image file",
        "content": {
            "application/octet-stream": {
                "schema": {
                    "type": "string",
                    "format": "binary",
                    "x-zen-display-name": "File Content"
                }
            }
        }
    }
}
\end{lstlisting}
\end{minipage}